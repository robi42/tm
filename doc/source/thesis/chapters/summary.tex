This thesis explored applying \gls{va} to \gls{datawrangling}, focusing on time-oriented\index{time-oriented} data.

Hence, after deepened study, presentation, and analysis\index{analysis} of related state of the art, an approach\index{approach} has been designed\index{design} and prototypically\index{prototype} implemented.
\gls{ux} personas and \gls{ui} mockups were valuable tools for our design\index{design} process.
Iteratively developing the prototype\index{prototype} in an agile manner was worthwhile as well.
As evaluation showed, there are still some issues mainly relating to usability, which can be further improved.
Nonetheless, the approach\index{approach} proved to be overall useful.

Our prototype\index{prototype} mainly offers interactive dashboard visualizations as a web-based application.
Special emphasis was put on crafting the \gls{ui} as well as applying \gls{va} methods reasonably.
Therefore, we have built visual-interactive\index{visual-interactive} charts supporting time-oriented data transformations\index{transformation}.
To make this all work well, we have also invested considerable effort in a sound underlying software design\index{design} and architecture.

Future work should focus on extending the amount of available transformations\index{transformation} supported via such visually interactive charting aids.
Specifically, giving the user more fine-grained control in some of the already present operations, plus adding completely new ones.
Also, additional attention should be paid to addressing scalability issues of the visualizations.
Particularly, in terms of data granularity as well as related coverage.
Moreover, enabling undo of operations, plus repetition of them via some sort of user-controlled history mechanism and/or storable scripts, somewhat similar to how DataWrangler does it, would be a reasonable addition.

Also, the inference aspect of the approach\index{approach}, interactively providing transform\index{transformation} suggestions, could be emphasized more and, thus, enhanced.
The outlier detection component we have integrated in our approach\index{approach} was generally well received in evaluation and tests, indicating this is leading into the right direction.
Moreover, as it is currently just a quite basic way of applying \gls{ml} techniques to the problem, there is definitely room for more.

While conducting the qualitative evaluation of our prototype quite some positive statements regarding the overall visual design, smooth \gls{ux}, and general quality of implemented interactive visualizations were made.
Things like \emph{``good overview''}, \emph{``can be easily done''}, and \emph{``it does what it's supposed to do''} come to mind.
Consequently, it appears our approach and prototype was generally perceived as well done and valuable.

In our opinion, wrangling\index{wrangle} time-oriented\index{time-oriented} data being supported by \gls{va} methods is an interesting field of research where \emph{TempMunger} just scratched the surface, delivering some input and, hopefully, inspiration to advance it further.
